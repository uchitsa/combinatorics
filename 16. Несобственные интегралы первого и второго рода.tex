\documentclass{article}
\usepackage{amsmath}

\begin{document}

\section{Несобственные интегралы первого рода}

Несобственный интеграл первого рода возникает, когда один или оба предела интегрирования бесконечны. Например:

\[
\int_{a}^{\infty} f(x) \, dx
\]

или

\[
\int_{-\infty}^{b} f(x) \, dx
\]

или

\[
\int_{-\infty}^{\infty} f(x) \, dx
\]

\section{Несобственные интегралы второго рода}

Несобственный интеграл второго рода возникает, когда подынтегральная функция имеет особенность (например, стремится к бесконечности) в пределах интегрирования. Например:

\[
\int_{a}^{b} f(x) \, dx
\]

где \( f(x) \) имеет особенность в точке \( c \) внутри интервала \([a, b]\).

\end{document}
